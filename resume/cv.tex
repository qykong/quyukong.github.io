\documentclass[a4paper,10pt]{article}

%A Few Useful Packages
\usepackage{marvosym}
\usepackage{fontspec} 					%for loading fonts
\usepackage{xunicode,xltxtra,url,parskip} 	%other packages for formatting
\RequirePackage{color,graphicx}
\usepackage[usenames,dvipsnames]{xcolor}
\usepackage[big]{layaureo} 				%better formatting of the A4 page
% an alternative to Layaureo can be ** \usepackage{fullpage} **
\usepackage{supertabular} 				%for Grades
\usepackage{titlesec}					%custom \section
\usepackage[ampersand]{easylist}
\usepackage{booktabs}% http://ctan.org/pkg/booktabs
\newcommand{\tabitem}{~~\llap{\textbullet}~~}
\usepackage{geometry}
 \geometry{
 a4paper,
 total={170mm,257mm},
 top=20mm,
 bottom=20mm
 }
%Setup hyperref package, and colours for links
\usepackage{hyperref}
\definecolor{linkcolour}{rgb}{0,0.2,0.6}
\hypersetup{colorlinks,breaklinks,urlcolor=linkcolour, linkcolor=linkcolour}
\usepackage{enumitem}
%FONTS
\defaultfontfeatures{Mapping=tex-text}
%\setmainfont[SmallCapsFont = Fontin SmallCaps]{Fontin}
%%% modified for Karol Kozioł for ShareLaTeX use
\setmainfont[
SmallCapsFont = Fontin-SmallCaps.otf,
BoldFont = Fontin-Bold.otf,
ItalicFont = Fontin-Italic.otf
]
{Fontin.otf}
%%%

%CV Sections inspired by: 
%http://stefano.italians.nl/archives/26
\titleformat{\section}{\Large\scshape\raggedright}{}{0em}{}[\titlerule]
\titlespacing{\section}{0pt}{3pt}{3pt}
%Tweak a bit the top margin
%\addtolength{\voffset}{-1.3cm}

%Italian hyphenation for the word: ''corporations''
\hyphenation{im-pre-se}

%-------------WATERMARK TEST [**not part of a CV**]---------------
\usepackage[absolute]{textpos}

\setlength{\TPHorizModule}{30mm}
\setlength{\TPVertModule}{\TPHorizModule}
\textblockorigin{2mm}{0.65\paperheight}
\setlength{\parindent}{0pt}

%--------------------BEGIN DOCUMENT----------------------
\begin{document}

\pagestyle{empty} % non-numbered pages

\font\fb=''[cmr10]'' %for use with \LaTeX command

%--------------------TITLE-------------
\par{\centering
		{\Huge Quyu \textsc{Kong}
	}\bigskip\par}

%--------------------SECTIONS-----------------------------------
%Section: Personal Data
\section{Contact Information}

\begin{tabular}{rl}
    \textsc{Address:}   & Hanna Neumann building, North Rd, ANU, Canberra, Australia \\
    \textsc{Phone:}     & +61 411559267\\
    \textsc{Email:}     & \href{mailto:quyu.kong@anu.edu.au}{quyu.kong@anu.edu.au} \\
    \textsc{Github:} &
    \href{https://github.com/qykong}{https://github.com/qykong} \\
    \textsc{Google Scholar:} &
    \href{https://scholar.google.com.au/citations?user=0EXa6lkAAAAJ&hl=en}{https://scholar.google.com.au/citations?user=0EXa6lkAAAAJ} \\
\end{tabular}

\section{Research Interests}
\textit{Information diffusion modeling in social media:} stochastic modeling, online networks, computational social socience, Hawkes processes, epidemic models
%Section: Work Experience at the top
\section{Education}
\begin{tabular}{r|p{11cm}}
 \textsc{Present} & \textsc{The Australian National University} \\
 \textsc{- Mar 2018}&\emph{PhD candidate in Computer Science, affiliated with Data61, CSIRO. }\\
 &\footnotesize{Advised by \href{http://rizoiu.eu/}{Dr. Marian-Andrei Rizoiu}, \href{http://users.cecs.anu.edu.au/~xlx/}{Prof. Lexing Xie}, and \href{https://people.csiro.au/W/S/Stephen-Wan}{Dr. Stephen Wan}} \\
 &\footnotesize{Thesis Proposed Topic: \textit{Linking Epidemic Models and Hawkes Point Processes for Modeling Information Diffusion}} \\
 \multicolumn{2}{c}{} \\
 
 \textsc{Dec 2017} & \textsc{The Australian National University} \\\textsc{- Feb 2016}&\emph{Advanced Master of Computing}\\&\footnotesize{Advised by \href{http://rizoiu.eu/}{Dr. Marian-Andrei Rizoiu}} \\
 &\footnotesize{Research Topic: \textit{Modeling Information Diffusion in Social Network}} \\
 \multicolumn{2}{c}{} \\
 
 \textsc{Aug 2011} & \emph{Zhejiang University} \\ \textsc{- Jun 2015} & \emph{Bachelor of Agronomy} \\\multicolumn{2}{c}{} \\
\end{tabular}


% \section{Projects}
% \begin{tabular}{lcc}
% \tabitem \emph{Hipdemo:}\\
% \hspace{1em} which is deployed on \href{www.hipdemo.ml}{hipdemo.ml}.\\
% \hspace{1em} \emph{Highlights:} Stochastic Process, Efficient Machine Learning Algorithm Implementation,\\ \hspace{1em} Interactive Visualisation
% \vspace{1em}
% \\

% \tabitem \emph{Modeling information diffusion in social network (One-year research project)} \\

% \hspace{1em} \emph{Highlights:} AMPL (large-scale optimisation), Spatial-temporal Point Process, Simulation \\
% \hspace{1em} and Prediction\\
% \\
% \end{tabular}

\section{Publications}
\begin{enumerate}[label={[\arabic*]}]
    \item \textbf{Kong, Quyu}, Marian-Andrei Rizoiu, Lexing Xie. ``Exploiting Uncertainty in Popularity Prediction of Information Diffusion Cascades Using Self-exciting Point Processes'' \textit{Under review. 2020.}
     \item \textbf{Kong, Quyu*}, Rohit Ram*, Marian-Andrei Rizoiu. ``A Toolkit for Analyzing and Visualizing Online Users via Reshare Cascade Modeling'' \textit{Under review. 2020.}
    \item \textbf{Kong, Quyu}, Marian-Andrei Rizoiu, Lexing Xie. ``Modeling Information Cascades with Self-exciting Processes via Generalized Epidemic Models" \textit{In Proceedings of the Thirteenth ACM International Conference on Web Search and Data Mining (WSDM). 2020.}
    \item \textbf{Kong, Quyu}. ``Linking Epidemic Models and Hawkes Point Processes for Modeling Information Diffusion." \textit{In Proceedings of the Twelfth ACM International Conference on Web Search and Data Mining (WSDM). 2019.}
    \item Rizoiu, Marian-Andrei, Swapnil Mishra, \textbf{Quyu Kong}, Mark Carman, and Lexing Xie. ``SIR-Hawkes: Linking Epidemic Models and Hawkes Processes to Model Diffusions in Finite Populations." \textit{In Proceedings of the Web Conference (WWW). 2018.}
    \item \textbf{Kong, Quyu}, Marian-Andrei Rizoiu, Siqi Wu, and Lexing Xie. ``Will This Video Go Viral: Explaining and Predicting the Popularity of Youtube Videos." \textit{In Companion Proceedings of the The Web Conference (WWW). 2018.}
\end{enumerate}

\section{Awards}
\begin{tabular}{rl}
\textsc{Nov 2019} & ANU CECS Dean's Travel Award \\
\textsc{Mar 2018} & Data61 PhD Scholarship \\
\textsc{Dec 2017} & ANU University Medal \\
\end{tabular}

\section{Teaching experience}
\begin{tabular}{r|p{11cm}}
  \textsc{Jun 2019} & \textit{Teaching Assistant at ANU} \\\textsc{- Jul 2017}&\footnotesize{COMP4650/COMP6490 Document Analysis, 2018}\\&\footnotesize{COMP4880/8880 Computational Methods for Network Science, 2019}
\end{tabular}

\section{Work experience}
\begin{tabular}{r|p{11cm}}
  \textsc{Feb 2020} & \textit{Part-time Research Assistant at ANU} \\\textsc{- Now}&\footnotesize{Developing an open source tool for modeling online information diffusions.}\\
  \multicolumn{2}{c}{} \\
  
 \textsc{Dec 2017} & \textit{Working Part-time at Spinify} \\\textsc{- Nov 2016}&\emph{Fullstack Developer}\\&\footnotesize{Building an interactive web app for visualising staff performance in workplace.}\\& \footnotesize{Highlights: web app, reactjs, nodejs, Amazon Lambda}\\\multicolumn{2}{c}{}\\
\end{tabular}

\section{Technical Skills}
\textbf{Programming Languages:} R (tidyverse), Python (Django, Flask), Javescript (React, Node.js)\\
\textbf{Web:} Fullstack development and maintenance (Computational Media Lab Webmaster)\\

\section{Open Source Projects}
\begin{tabular}{lcc}
\tabitem \textbf{evently:} A package designed for simulating and fitting the Hawkes processes and the \\
\hspace{1em} HawkesN processes with several options of kernel functions. Code is available at:\\
\hspace{1em}  \href{https://github.com/behavioral-ds/evently}{https://github.com/behavioral-ds/evently}\\
\multicolumn{2}{c}{}\\
\tabitem \textbf{Hip-demo:} An interactive web visualizer written in R language with Shiny library\\
\hspace{1em} which is deployed on \href{www.hipie.ml}{hipie.ml}. Code is available at: \href{https://github.com/qykong/hipdemo}{github.com/qykong/hipdemo}\\
% \hspace{1em} \emph{Highlights:} Stochastic Process, Efficient Machine Learning Algorithm Implementation,\\ \hspace{1em} Interactive Visualisation
\end{tabular}
\\
% \multicolumn{2}{c}{} \\
 
% \section{Hobbies \& Interests}

% %Section: Languages
% \section{Languages}
% \begin{tabular}{rl}
%  \textsc{Chinese:}&Mothertongue\\
% \textsc{English:}&Fluent\\
% \end{tabular}

\end{document}
